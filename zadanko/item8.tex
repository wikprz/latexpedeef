\begin{frame}{Windows 11}
Windows 11-  Najnowszy system operacyjny firmy Microsoft, wydany 5 października 2021 roku jako następca Windows 10.
 Charakteryzuje się odświeżonym wyglądem z wycentrowanym menu Start i zaokrąglonymi rogami okien, oferuje lepszą wydajność dzięki szybszemu uruchamianiu i bardziej efektywnemu zarządzaniu zasobami, wprowadza nowe funkcje takie jak Copilot (asystent AI), ulepszone narzędzia do wielozadaniowości, wsparcie dla aplikacji Android, poprawione możliwości gamingowe z Auto HDR i DirectStorage, zwiększone bezpieczeństwo dzięki wymaganiom TPM 2.0 i Secure Boot, a także lepszą integrację z usługami Microsoft, w tym Teams i Xbox Game Pass. Windows 11 kładzie nacisk na produktywność, kreatywność i bezpieczeństwo, oferując jednocześnie bardziej intuicyjne i spersonalizowane doświadczenie użytkownika. Jego wymagania sprzętowe są jednak bardziej „surowe” niż w przypadku jego poprzednika.Wymagania sprzętowe:- Procesor: 1 GHz lub szybszy z co najmniej 2 rdzeniami- RAM: minimum 4 GB- Pamięć: minimum 64 GB wolnego miejsca- Karta graficzna: zgodna z DirectX 12- Wyświetlacz: rozdzielczość HD (720p) lub wyższa- TPM 2.0 i obsługa Secure Boot






\end{frame}
